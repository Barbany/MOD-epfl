\documentclass[letterpaper]{article}
\usepackage[top=2cm, bottom=3.5cm, left=2.5cm, right=2.5cm]{geometry}
\usepackage{amsmath,amsthm,amssymb}
% Indicator function
\usepackage{bbm}
\usepackage{tikz}
\usepackage{enumitem}
\usepackage{booktabs}
\usepackage[T1]{fontenc}

% Bold vectors and matrices
\renewcommand{\aa}{\mathbf{a}}
\providecommand{\xx}{\mathbf{x}}
\providecommand{\yy}{\mathbf{y}}
\providecommand{\1}{\mathbf{1}}
\providecommand{\0}{\mathbf{0}}
\providecommand{\mA}{\mathbf{A}}
\providecommand{\mI}{\mathbf{I}}

\providecommand{\lin}[1]{\ensuremath{\left\langle #1 \right\rangle}}
\providecommand{\norm}[1]{\ensuremath{\left\lVert#1\right\rVert}}

%
% The following macro is used to generate the header.
%
\newcommand{\homework}[4]{
   \thispagestyle{plain}
   \newpage
   \noindent
   \begin{center}
   \framebox{
      \vbox{\vspace{2mm}
    \hbox to 6.28in { {\bf EE-556: Mathematics of Data \hfill Fall 2019} }
       \vspace{6mm}
       \hbox to 6.28in { {\Large \hfill Homework \##1 - Due date: #2\hfill} }
       \vspace{4mm}
       \hbox to 6.28in { {\hfill Student: #3} }
      \vspace{2mm}}
   }
   \end{center}
}

\renewcommand{\phi}{\varphi}
\renewcommand{\epsilon}{\varepsilon}

% Use these for theorems, lemmas, proofs, etc.
\newtheorem{proposition}{Proposition}
\newtheorem{theorem}{Theorem}
\newtheorem{corollary}{Corollary}
\newtheorem{claim}{Claim}
\newtheorem{remark}{Remark}
\newtheorem{definition}{Definition}
\newtheorem{fact}{Fact}
\newtheorem{assumption}{Assumption}

\DeclareMathOperator*{\argmin}{arg\,min}
\DeclareMathOperator*{\argmax}{arg\,max}
\newcommand{\mbeq}{\overset{!}{=}}
\newcommand{\E}{\mathbb{E}}
\newcommand{\vect}[1]{\boldsymbol{#1}}

% Use Small capitals like in problem statement
\usepackage{sectsty}
\renewcommand{\thesection}{\Roman{section}} 
\renewcommand{\thesubsection}{\Roman{subsection}}
\allsectionsfont{\mdseries\scshape}

\begin{document}
\homework{1}{1\textsuperscript{st} November 2019}{Oriol Barbany Mayor}

\subsection*{Problem 1 - Geometric properties of the objective function $f$}
\begin{enumerate}[label=(\alph*)]
    \item By linearity of the gradient operator, we have that
    \begin{align}
        f(\xx, \mu) := \frac{1}{n}\sum_{i=1}^n g_i (\xx, \mu) + \frac{\lambda}{2}\norm{\xx}^2 \Longrightarrow \nabla f(\xx) = \frac{1}{n}\sum_{i=1}^n \nabla g_i (\xx, \mu) + \lambda \xx
    \end{align}
    
    \begin{align}
        \nabla g_i (\xx, \mu) = \begin{cases}
        -b_i \aa_i, & b_i \aa_i^T \xx \leq 0\\
        b_i \aa _i (b_i \aa_i^T \xx - 1), &0<b_i\aa_i^T \xx \leq 1 \\
        0,& 1\leq b_i \aa_i^T \xx
        \end{cases}
        \label{eq:1}
    \end{align}
    which by noting that given that $b_i \in \{-1,+1\}$, $b_i^2 = 1 \ \forall i$ yields
    
    \begin{align}
        \nabla f(\xx) = \frac{1}{n}\sum_{i=1}^n \left[ -b_i \aa_i\mathbbm{1}_{\{b_i \aa_i^T \xx  \leq 0\}} + \aa _i (\aa_i^T \xx - b_i )\mathbbm{1}_{\{0<b_i \aa_i^T \xx\leq 1\}} \right] + \lambda \xx
        \label{eq:2}
    \end{align}
    
    Note that with the proposed notation, $\mI_L$ selects the coordinates where we are in the first case of \eqref{eq:1}, $\mI_Q$ the second case. If we express \eqref{eq:2} in matrix form using $\tilde{\mA}$ and the previous matrices, we get
    \begin{align}
        \nabla f(\xx) = - \frac{1}{n} \tilde{\mA} \mI_L \1 + \frac{1}{n} \tilde{\mA}^T \mI_Q [\tilde{\mA} \xx - \1] + \lambda \xx
    \end{align}
    
    %Assume that all samples lie in the linear part, i.e. $b_i (\aa_i^T \xx + \mu) \leq 0 \ \forall i \in [n] := \{1,\dots, n\}$ and hence $\mI_L = \mathbb{I}$, $\mI_Q = \0$. Then,
    %\begin{align}
    %    \norm{\nabla f(\xx) - \nabla f(\yy)} = \norm{\lambda (\xx - \yy)} = |\lambda|\norm{\xx - \yy} := L \norm{\xx - \yy}
    %\end{align}
    %which means that in the linear case and assuming $\lambda\geq0$, $f$ is $\lambda-$smooth.
    
    Assume that all the samples lie in the quadratic region
    \begin{align}
        \norm{\nabla f(\xx) - \nabla f(\yy)} &= \norm{\lambda (\xx - \yy) + \frac{1}{n}\tilde{\mA}^T \tilde{\mA} (\xx - \yy)} = \norm{\left( \lambda\mathbf{I} + \frac{1}{n}\tilde{\mA}^T \tilde{\mA} \right) (\xx - \yy)} \\
        &\leq \norm{\lambda\mathbf{I} + \frac{1}{n}\tilde{\mA}^T \tilde{\mA}} \norm{\xx - \yy}
    \end{align}
    where the inequality follows from definition of the spectral norm.
    
    The spectral norm satisfies the triangle inequality (as every norm), so 
    \begin{align}
        \norm{\lambda\mathbf{I} + \frac{1}{n}\tilde{\mA}^T \tilde{\mA}} &:= \max_{\xx:\norm{\xx}=1} \norm{\left( \lambda\mathbf{I} + \frac{1}{n}\tilde{\mA}^T \tilde{\mA} \right) \xx} \leq |\lambda| \max_{\xx:\norm{\xx}=1} \norm{\xx}  +\frac{1}{n} \max_{\xx:\norm{\xx}=1}  \norm{\tilde{\mA}^T \tilde{\mA}\xx} \\
        &:= \lambda + \frac{1}{n}\norm{\tilde{\mA}^T \tilde{\mA}} \leq \lambda + \frac{1}{n}\norm{\tilde{\mA}^T} \norm{\tilde{\mA}}
    \end{align}
    where the last inequality again follows from definition of the spectral norm. The proof is completed using the fact that $\norm{\tilde{\mA}}=\norm{\mA}$ and $\norm{\tilde{\mA}^T}=\norm{\mA^T}$.
    
    \item Assuming again that all the samples lie in the quadratic region,
    \begin{align}
        \nabla f(\xx) = \frac{1}{n} \tilde{\mA}^T [\tilde{\mA} \xx - \1] + \lambda \xx
    \end{align}
    the gradient $\nabla f$ exists in all domain of $\xx$ and correspond to a linear function, hence differentiable. We then say that $f$ is twice differentiable, and its hessian corresponds to
    \begin{align}
        \nabla^2 f(\xx) = \frac{1}{n} \tilde{\mA}^T \tilde{\mA} + \lambda\mathbb{I}=\frac{1}{n}\mA^T \mA + \lambda\mathbb{I}
    \end{align}
    where the equality follows since $b_i^2 = 1 \ \forall i$, which implies that $\tilde{\mA}^T \tilde{\mA} = \mA^T \mA$.
    
    \item 
    \begin{proposition}
        A function $f$ is $\mu-$strongly convex iff $\nabla^2 f(\xx) \succeq \mu \mathbb{I}$
        \label{prop:1}
    \end{proposition}
    \begin{proof}
        As seen in lectures, $f$ is $\mu-$strongly convex iff $g(\xx):=f(\xx)-\frac{\lambda}{2}\norm{\xx}$ is convex, and hence if $\nabla^2g(\xx) \succeq 0$. The proposition follows by combining these two.
    \end{proof}
    
    Given that we have the hessian of $f$, we can use Proposition \ref{prop:1} to compute the strong convexity constant.
    \begin{align}
        \xx^T \nabla^2 f(\xx) \xx  = \lambda + \frac{1}{n}\xx^T\mA^T \mA \xx \geq \lambda := \mu \quad \forall \xx
    \end{align}
    where the inequality follows since $\mA^T \mA$ is positive semidefinite. This latter is true since we can define $\yy:=\mA \xx$ and $\norm{y} := \yy^T \yy := \xx^T\mA^T \mA \xx $ will be non-negative by definition of the norm.
\end{enumerate}

\subsection*{Problem 3 - Stochastic Gradient methods for SVM}
\begin{enumerate}[label=(\alph*)]
    \item
    Since we choose each index $i_k$ uniformly at random from all the $n$ samples, we have that
    \begin{align}
        \E[\nabla f_{i_k}(\xx)] &= \frac{1}{n} \sum_{i=1}^{n} \nabla f_i(\xx) \\
        &= \lambda \xx + \frac{1}{n}\sum_{i=1}^n \left[ -b_i \aa_i\mathbbm{1}_{\{b_i \aa_i^T \xx  \leq 0\}} + \aa _i(\aa_i^T \xx - b_i )\mathbbm{1}_{\{0<b_i \aa_i^T \xx\leq 1\}} \right] \\
        &:= \nabla f(\xx)
    \end{align}
    where the last equality is the one already found in \eqref{eq:1}.
    
    \item Assume that sample $i_k$ lies in the quadratic region
    \begin{align}
        \norm{\nabla f_{i_k}(\xx) - \nabla f_{i_k}(\yy)} = \norm{\lambda(\xx - \yy) + \aa_{i_k} \aa_{i_k}^T (\xx - \yy)} \leq \norm{\lambda\mathbb{I} + \aa_{i_k} \aa_{i_k}^T}\norm{\xx + \yy}
    \end{align}
    where the inequality follows by definition of the spectral norm.
    
    Using triangle inequality as in problem 1 gives
    \begin{align}
        \norm{\lambda\mathbb{I} + \aa_{i_k} \aa_{i_k}^T} &\leq \lambda + \norm{\aa_{i_k} \aa_{i_k}^T} := \lambda + \max_{\xx:\norm{\xx}=1} \norm{\aa_{i_k} \aa_{i_k}^T x} := \lambda + \max_{\xx:\norm{\xx}=1} \sqrt{x^T \aa_{i_k} \aa_{i_k} ^T \aa_{i_k} \aa_{i_k}^T x} \\
        &:=\lambda + \max_{\xx:\norm{\xx}=1} \sqrt{x^T \norm{\aa_{i_k}} ^2 \norm{\aa_{i_k}} ^2 x} = \lambda + \norm{\aa_{i_k}} ^2 \max_{\xx:\norm{\xx}=1} \norm{x} = \lambda + \norm{\aa_{i_k}} ^2
    \end{align}
    
\end{enumerate}

\end{document}
